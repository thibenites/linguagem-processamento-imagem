\documentclass[a4paper, 10pt, conference]{ieeeconf}

\IEEEoverridecommandlockouts
\overrideIEEEmargins

\usepackage{graphics} % for pdf, bitmapped graphics files
\usepackage{epsfig} % for postscript graphics files
\usepackage{mathptmx} % assumes new font selection scheme installed
\usepackage{times} % assumes new font selection scheme installed
\usepackage{amsmath} % assumes amsmath package installed
\usepackage{amssymb}  % assumes amsmath package installed
\usepackage{hyperref}
\usepackage{listings}
\usepackage{subfigure}

\usepackage[T1]{fontenc}
\usepackage[utf8]{inputenc}
\usepackage[brazil]{babel}
\usepackage{etoolbox}
\patchcmd{\abstract}{Abstract}{Resumo}{}{}
\patchcmd{\thebibliography}{References}{Referências}{}{}

\title{\LARGE \bf Otimizando o tempo de execução no processamento de imagens}

\author{Henrique Miyamoto e Thiago Benites \thanks{* Os códigos do projeto estão disponíveis em \url{https://github.com/miyamotohk/linguagem-processamento-imagem}.}}

\begin{document}
\maketitle
\thispagestyle{empty}
\pagestyle{empty}

%\begin{abstract}

%Escreva aqui o resumo (abstract).

%\end{abstract}

\section{Contextualização}

%Um breve texto introdutório explicando do que se trata o documento, em uma linguagem que poderia ser entendida por qualquer pessoa que entenda programação (ou seja: referências diretas à disciplina não são desejáveis).

Apresentamos uma linguagem de programação com diferentes maneiras de realizar o processamento de imagens digitais. Ela foi implementada com uma gramática livre de contexto, usando analisadores léxicos e sintáticos. Como uma linguagem de propósito específico, pretende-se que seja intuitiva para o usuário \cite{dsl}. As funcionalidades que nossa linguagem é capaz de executar, assim como as respectivas sintaxes são apresentadas na Tabela \ref{tabela1}.

\begin{table}[h]
	\centering
	\caption{Funcionalidades e sintaxe da linguagem de programação}
	\label{tabela1}
	\begin{tabular}{|c|c|}
		\hline
		\textbf{Funcionalidade}        & \textbf{Sintaxe}                                                                                                                \\ \hline
		Salvar uma imagem     & {destino.jpg = origem.jpg}                                                                                    \\ \hline
		Alterar brilho        & \begin{tabular}[c]{@{}c@{}}{destino.jpg = origem.jpg * float}\\ {destino.jpg = origem.jpg / float}\end{tabular} \\ \hline
		Alterar brilho usando threads        & \begin{tabular}[c]{@{}c@{}}{destino.jpg = origem.jpg * float thr}\end{tabular} \\ \hline
		Alterar brilho usando processos        & \begin{tabular}[c]{@{}c@{}}{destino.jpg = origem.jpg * float prc}\end{tabular} \\ \hline
		Detectar valor máximo & {[origem.jpg]}                                                                                            \\ \hline
	\end{tabular}
\end{table}

Para as funções de alterar o brilho, foram realizadas medições temporais do desempenho. Sendo possível haver uma comparação entre os tempos, a discussão de otimização baseia-se nesses tempos e em suas respectivas implementações.

Neste caso, a implementação simples se dá pela varredura, pixel a pixel, da matriz que representa a imagem. Já a implementção com multithread, faz com que as operações de alteração do brilho sejam feitas em grupos de pixel e de forma paralela.

\section{Demonstração}

%Entradas e saídas que demonstram as funcionalidades implementadas.

A tabela \ref{tabela2} indica os tempos de processamento para a aplicação da mesma intensidade de brilho em imagens de diferentes tamanhos. Ademais, alterou-se a quantidade de pixels analisadas a cada vez que uma função multithread foi chamada.

\begin{table}[h]
	\centering
	\caption{Tempo de processamento de acordo com a implementação e tamanho da imagem}
	\label{tabela2}
	\begin{tabular}{|c|c|c|}
		\hline
		\textbf{Implementação}        & \textbf{Tamanho}                                                                                                               & \textbf{Tempo} \\ \hline
		Simples     & {48x48}	& {0,046 ms}                                                                                    \\ \hline
		Multithread (10 pixels)        & {48x48}	& {3,31 ms} \\ \hline
		Multithread (48 pixels)        & {48x48}	& {0,45 ms} \\ \hline
		Simples     & {2592x1944}	& {136,63 ms}                                                                                    \\ \hline
		Multithread (600 pixels)        & {2592x1944}	& {69,87 ms} \\ \hline
		Multithread (2592 pixels)        & {2592x1944}	& {14,99 ms} \\ \hline
		
	\end{tabular}
\end{table}

\section{Análise}

%Aqui deve-se comparar a qualidade, em questão de tempo, de cada implementação. Ademais, analisar-se-á, quais métodos são vantajosos de acordo com diferentes tamanhos de imagens e de pixels fixados para análise por função.



\begin{thebibliography}{99}

\bibitem{dsl} MERNIK, M., HEERIN, J., SLOANE, A. M. When and how to develop domain-specific languages. In: \textit{ACM Computing Surveys (CSUR)}. Nova York, vol. 37, ed. 4, p. 316-344, dez. 2005.

\end{thebibliography}

\end{document}